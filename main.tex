%%%%%%%%%%%%%%%%%%%%%%%%%%%%%%%%%%%%%%%%%
%
% Developer CV
% LaTeX Template
% Version 1.0 (28/1/19)
%
% This template originates from:
% http://www.LaTeXTemplates.com
%
% Authors:
% Jan Vorisek (jan@vorisek.me)
% Based on a template by Jan Küster (info@jankuester.com)
% Modified for LaTeX Templates by Vel (vel@LaTeXTemplates.com)
%
% License:
% The MIT License (see included LICENSE file)
%
%%%%%%%%%%%%%%%%%%%%%%%%%%%%%%%%%%%%%%%%%

%----------------------------------------------------------------------------------------
%	PACKAGES AND OTHER DOCUMENT CONFIGURATIONS
%----------------------------------------------------------------------------------------

\documentclass[9pt]{developercv} % Default font size, values from 8-12pt are recommended

%----------------------------------------------------------------------------------------

\begin{document}

%----------------------------------------------------------------------------------------
%	TITLE AND CONTACT INFORMATION
%----------------------------------------------------------------------------------------

\begin{minipage}[t]{0.45\textwidth} % 45% of the page width for name
    \vspace{-\baselineskip} % Required for vertically aligning minipages

    % If your name is very short, use just one of the lines below
    % If your name is very long, reduce the font size or make the minipage wider and reduce the others proportionately
    \colorbox{black}{{\HUGE\textcolor{white}{\textbf{\MakeUppercase{Artem}}}}} % First name

    \colorbox{black}{{\HUGE\textcolor{white}{\textbf{\MakeUppercase{Vanyukhin}}}}} % Last name

    \vspace{6pt}

    {\huge Full Stack Software Engineer } % Career or current job title
\end{minipage}
\begin{minipage}[t]{0.275\textwidth} % 27.5% of the page width for the first row of icons
    \vspace{-\baselineskip} % Required for vertically aligning minipages

    % The first parameter is the FontAwesome icon name, the second is the box size and the third is the text
    % Other icons can be found by referring to fontawesome.pdf (supplied with the template) and using the word after \fa in the command for the icon you want
    \icon{MapMarker}{12}{Prague, CZ}\\
    \icon{Phone}{12}{+420 773 630 896}\\
    \icon{At}{12}{\href{mailto:v96artem@gmail.com}{v96artem@gmail.com}}\\
\end{minipage}
\begin{minipage}[t]{0.275\textwidth} % 27.5% of the page width for the second row of icons
    \vspace{-\baselineskip} % Required for vertically aligning minipages

    % The first parameter is the FontAwesome icon name, the second is the box size and the third is the text
    % Other icons can be found by referring to fontawesome.pdf (supplied with the template) and using the word after \fa in the command for the icon you want
    \icon{LinkedinSquare}{12}{\href{https://www.linkedin.com/in/artem-vanyukhin/}{artem.vanyukhin}}\\
    \icon{Github}{12}{\href{https://github.com/vanyuart}{github.com/vanyuart}}\\
\end{minipage}

\vspace{0.5cm}

%----------------------------------------------------------------------------------------
%	INTRODUCTION, SKILLS AND TECHNOLOGIES
%----------------------------------------------------------------------------------------

\cvsect{Who Am I?}

\begin{minipage}[t]{0.27\textwidth} % 40% of the page width for the introduction text
    \vspace{-\baselineskip} % Required for vertically aligning minipages
    Avid learner with an imperative for challenging tasks.
    Open to responsibilities.
    Adaptable to various work environments.
\end{minipage}
\hfill % Whitespace between
\begin{minipage}[t]{0.73\textwidth} % 50% of the page for the skills bar chart
    \vspace{-\baselineskip} % Required for vertically aligning minipages
    \begin{barchart}{5.5}
        \baritem{Java}{100}
        \baritem{Kotlin}{90}
        \baritem{JavaScript (React, Knockout, Vanilla)}{95}
        \baritem{TypeScript}{45}
        \baritem{Node.js}{30}
        \baritem{C\#}{65}
        \baritem{.NET}{45}
        \baritem{SQL (Oracle, Postgres, MySQL)}{85}
        \baritem{NoSQL (MongoDB)}{70}
    \end{barchart}
\end{minipage}

%----------------------------------------------------------------------------------------
%	EXPERIENCE
%----------------------------------------------------------------------------------------

\cvsect{Experience}

\begin{entrylist}
    \entry
    {2/2020 -- Present}
    {Backend developer}
    {ČSOB via Neon IT}
    {
        Joined ČSOB as an external employee via NeonIT.
        I am a member of the development team on several projects.\\

        \textbf{CIM (Client Identity Management)}\\
        Internal system used by the majority of ČSOB applications.
        Manages clients personal data, their accounts' information, available authorization methods, signed documents etc.\\

        \textbf{BankID}\\
        Access to Citizen Portal via ČSOB identity.\\

        \textbf{CAT (Credits Approvals Transactions)}\\
        Management of cards' limits, account access authorization etc.
        This project was written purely in Kotlin.\\

        \texttt{Java}\slashsep
        \texttt{Kotlin}\slashsep
        \texttt{Spring}\slashsep
        \texttt{OracleSQL}\slashsep
        \texttt{RabbitMQ}\slashsep
        \texttt{Kafka}\\
    }
    \entry
    {2/2017 -- 1/2020}
    {Full stack Java developer}
    {Omax Holding}
    {
        Started at Omax as a member of the development team on NaseStravenka.
        Later was assigned to manage the development of a CRM subsystem.
        I had to meet with the client and was responsible for requirements collection.
        Closer to the final stage of the project I started learning .NET and was assigned to ProductAds project.\\

        \textbf{NaseStravenka (Lidl)}\\
        Project consisted of multiple systems: REST API, web for clients, internal portal for employees, CMS etc.
        At the later stage I was assigned to develop CRM subsystem for the portal.

        \texttt{Java}\slashsep
        \texttt{Spring}\slashsep
        \texttt{Hibernate}\slashsep
        \texttt{Oracle SQL}\slashsep
        \texttt{Knockout.js}\slashsep
        \texttt{Thymeleaf}\\

        \textbf{ProductAds (Heureka)}\\
        ProductAds is a system for ads auctions and management on Heureka (CZ/SK/HU).
        My main task was to set up dockerized ELK stack for monitoring and to learn .NET as I go.

        \texttt{.NET}\slashsep
        \texttt{MongoDB}\slashsep
        \texttt{MySQL}\slashsep
        \texttt{Docker}\slashsep
        \texttt{ELK}\slashsep
        \texttt{Prometheus}\slashsep
        \texttt{Grafana}\\
    }
\end{entrylist}

\newpage

%----------------------------------------------------------------------------------------
%	EDUCATION
%----------------------------------------------------------------------------------------

\cvsect{Education}

\begin{entrylist}
    \entry
    {2019 -- 2021}
    {Bachelor's Degree}
    {CTU In Prague, FEL}
    {
        I am finishing my Bc. degree alongside a full time job.
        My bachelor's thesis is a web application for personal development written in Kotlin with Spring on backend and React.js with Material UI on frontend.
    }

    \entry
    {2016 -- 2018}
    {Bachelor's Degree}
    {CTU In Prague, FEL}
    {
        I had to terminate studies on my 3rd year due to career opportunity.
        At the university I had the opportunity to learn C++ which deepened my overall knowledge of CS.
    }
\end{entrylist}

%----------------------------------------------------------------------------------------
%	ADDITIONAL INFORMATION
%----------------------------------------------------------------------------------------

\begin{minipage}[t]{0.3\textwidth}
    \vspace{-\baselineskip} % Required for vertically aligning minipages

    \cvsect{Languages}

    \textbf{Russian} - native\\
    \textbf{Czech} - advanced\\
    \textbf{English} - advanced
\end{minipage}
\hfill
\begin{minipage}[t]{0.7\textwidth}
    \vspace{-\baselineskip} % Required for vertically aligning minipages

    \cvsect{Hobbies}

    These are just few of my many hobbies.
    I enjoy cooking, especially baking.
    I am passionate about music and I play several musical instruments.

\end{minipage}
\hfill

%----------------------------------------------------------------------------------------

\end{document}
